\documentclass{article}
\usepackage{amsmath}

\begin{document}
	6.2
	$$ H = \omega(q^2+p^2)/2$$
	$$ f = \frac{\partial{H}}{\partial{p}} = \omega p$$
	$$ g = -\frac{\partial{H}}{\partial{q}} = -\omega q$$
	Define new coordinates:
	$$ q = r cos\phi$$
	$$ p = -rsin\phi$$
	The Jacobian of this transformation is:
	\begin{multline*}
		J =\begin{Bmatrix}
			\frac{\partial{q}}{\partial{r}}&\frac{\partial{q}}{\partial{\phi}}\\
			\frac{\partial{p}}{\partial{r}}&\frac{\partial{p}}{\partial{\phi}}\\
			\end{Bmatrix} 
			= \begin{Bmatrix}
			cos\phi & -rsin\phi\\
			-sin\phi & -rcos\phi\\
			\end{Bmatrix} 
			=>
			J^{-1} = \frac{1}{r}\begin{Bmatrix}
			rcos\phi & -rsin\phi\\
			-sin\phi & -cos\phi\\
			\end{Bmatrix} 
			= \begin{Bmatrix}
			\frac{\partial{r}}{\partial{q}}&\frac{\partial{r}}{\partial{p}}\\
			\frac{\partial{\phi}}{\partial{q}}&\frac{\partial{\phi}}{\partial{p}}\\
			\end{Bmatrix} 
	\end{multline*}
	Plug this into the Vlasov Equation:
	\begin{equation*}
		\begin{split}
		VE &= \frac{\partial{\Psi}}{\partial{t}}+f\frac{\partial{\Psi}}{\partial{q}}+q\frac{\partial{\Psi}}{\partial{p}}\\
		&= \frac{\partial{\Psi}}{\partial{t}}
		+\omega p\left[\frac{\partial{\Psi}}{\partial{r}}\cos{\phi}-\frac{\partial{\Psi}}{\partial{\phi}}\frac{\sin{\phi}}{r}\right]
		-\omega q \left[\frac{\partial{\Psi}}{\partial{r}}\sin{\phi}-\frac{\partial{\Psi}}{\partial{\phi}}\frac{\cos{\phi}}{r}\right]\\
		&= \frac{\partial{\Psi}}{\partial{t}}+
		\omega\frac{\partial{\Psi}}{\partial{r}}\left(-r\sin{\phi}\cos{\phi}+r\cos{\phi}\sin{\phi}\right)
		+\omega\frac{\partial{\Psi}}{\partial{\phi}}\left(r\sin{\phi}\sin{\phi}/r+r\cos{\phi}\cos{\phi}/r\right) \\
		&= \frac{\partial{\Psi}}{\partial{t}}+\omega\frac{\partial{\Psi}}{\partial{\phi}}=0 
		\end{split}
	\end{equation*}
\end{document}
